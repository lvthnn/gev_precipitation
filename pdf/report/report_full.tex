\documentclass[12pt, a4paper]{paper}

\usepackage[T1]{fontenc}
\usepackage[utf8]{inputenc}

\usepackage[scale = 0.75]{geometry}

\usepackage[UKenglish]{babel}

\usepackage[scale = 0.9]{inter}
\usepackage{newtx}
\usepackage{amsmath, amsfonts}
\usepackage[scaled = 0.8]{beramono}

\usepackage{graphicx}
\usepackage{tabularx}
\usepackage{caption}
\usepackage{cleveref}
\usepackage{float}

\usepackage{enumitem}

\usepackage[sorting = none]{biblatex}
\usepackage{csquotes} \addbibresource{references.bib}

\graphicspath{{../img}}

\usepackage{booktabs}

\captionsetup{
  labelfont = {bf, sf},
  labelsep = period,
  font = small
}

\title{Application of Bayesian methods to modelling of extreme precipitation
        events}
\author{Kári Hlynsson}
\date{}

\begin{document}
  
  \maketitle

  \section{Introduction}
  Extreme precipitation events, such as flooding, pose significant risks to
  both infrastructure and human safety.\@ Events are unpredictable, localised,
  and difficult to estimate due to sparse data, which are sourced from gauging
  stations built in close proximity to catchments such as streams, wells, lakes,
  canals, and reservoirs.\@ \cite{dyrrdal_bayesian_2015} Accurate modelling and
  prediction is vital for the implementation of preventive measures to mitigate
  damage caused by such events, as well as serving as interesting applications
  a host of statistical methods explored here.

  In this article, the application of Bayesian modelling methods, particularly
  Markov-chain Monte Carlo sampling, to annual maximum precipitation data
  totalling \(N = 132\) measurements is considered.\@ The generalised extreme
  value (GEV) distribution is used to model the response layer with (i) fixed
  parameters, or a "stationary" model, and (ii) an evolving location parameter
  with respect to time, a "non-stationary" model.\@ For the latter type, two
  models are proposed; the former implements a linear time trend, whereas the
  latter model allows a more flexible trend.\@ Two information criteria,
  WAIC and LOOIC, are used for model comparison and selection.\@ Posterior
  predictive checks are used for model evaluation. Data processing and
  visualisation was performed in the \texttt{R} language and MCMC
  sampling was implemented in \texttt{Stan}. \cite{carpenter_stan_2017,
  r_core_team_r_2024} Code and data used in the project are available at
  \url{https://github.com/lvthnn/gev_precipitation_data}.

  \section{Preliminaries}
  \subsection{Theoretical background}
  The generalised extreme value (GEV) distribution has parameters
  \(\mu \in \mathbf R\), \(\sigma \in \mathbf R_+\), and \(\xi \in \mathbf R\),
  termed the location, shape, and scale parameters, respectively. The cumulative
  density is
  \[
    \mathrm{GEV}(y \mid \mu, \sigma, \xi) =
    \begin{cases}
      \exp\left(-\big[1 + \xi (y - \mu)/\sigma\big]^
      {-1/\xi}_+\right), &\xi \neq 0, \\
      \exp\left(-\exp\left[-(y - \mu)/\sigma\right]
      \right), &\xi = 0
    \end{cases}
  \]
  and support \((-\infty, \mu - \sigma/\xi]\), \((-\infty, \infty)\), and
  \([\mu - \sigma/\xi, \infty)\) for \(\xi < 0\), \(\xi = 0\), and \(\xi > 0\),
  respectively. This distribution arises naturally in a statistical setting
  involving normalised block maxima of
  random variables \((X_n)_{n \in \mathbf N}\), defined by \((M_n - a_n)/b_n\)
  for suitable constants \(a_n \in \mathbf R_+\) and \(b_n \in \mathbf R\)
  where \(M_n = \max\{X_1, \ldots, X_n\}\), as by the Fisher-Tippett theorem,
  the only limiting distribution for \(M_n\) is the GEV distribution.
  \cite{coles_introduction_2001,hazra_bayesian_2023} In other words, if
  \[
    \Pr\{(M_n - b_n) / a_n \leq y\} \to G(y),
  \]
  then necessarily \(G(y) = \mathrm{GEV}(y \mid \mu, \sigma, \xi)\) for
  suitable \(\mu \in \mathbf R\), \(\sigma \in \mathbf R_+\), and \(\xi \in
  \mathbf R\). If \(Y\) is a GEV random variable, then it has expectation
  \[
    \mathrm{E}[Y] = \begin{cases}
      \mu + \sigma[\Gamma(1 - \xi) - 1]/\xi
        &\xi \in (-\infty, 1) \setminus \{0\}, \\
      \mu + \sigma\gamma &\xi = 0, \\
      \infty &\xi \geq [1, \infty),
    \end{cases}
  \]
  and variance given by
  \[
    \mathrm{var}[Y] = \begin{cases}
      \sigma^2 [\Gamma(1 - 2\xi) - \Gamma(1 - \xi)^2] / \xi^2
        & \xi \in (-\infty, 1/2) \setminus \{0\}, \\
      \sigma^2 \pi^2 / 6, & \xi = 0, \\
      \infty & \xi \geq 1/2,
    \end{cases}
  \]
  where \(\Gamma(\cdot)\) is the gamma function and \(\gamma \approx 0.5772\)
  is Euler's constant.

  \subsection{Converge diagnostics, model selection, and evaluation}
  For each model, four separate chains are run for a total of 9000 iterations
  and the first 1000 are discarded as burn-in. To ensure that the chains are
  sufficiently close to stationarity, we use trace and autocorrelation plots for
  visual confirmation of convergence, and the potential scale reduction factor,
  denoted \(\hat R\), as well effective sample size \(\widehat{\mathrm{ESS}}\).
  The former diagnostic compares parameter estimates between and within the
  different chains respectively. If the estimates are very similar, \(\hat R\)
  is close to one.\@ Higher values may indicate that chains run simultaneously
  from different initial values have not converged and may thus indicate a
  problem with the sampler.

  The latter quantity, the effective sample size, or ESS, is defined as
  \[
    \mathrm{ESS} = \frac{N_{\mathrm{iter}}}{1 + 2 \sum_{l = 1}^\infty \rho(l)},
  \]
  where \(N_\mathrm{iter}\) is the number of samples drawn, and \(\rho(l)\) is
  the autocorrelation between samples \(l\) iterations apart. Low effective
  sample size indicates that samples produced by the MCMC algorithm are
  highly correlated and thus provide a biased view of the posterior
  distribution. \cite{stan_development_team_convergence_nodate} Resultingly,
  parameter inference is unreliable.

  WAIC and LOOIC are used to compare models and to select
  the optimal value of the hyperparameter \(m\) in the case of piecewise
  hierarchical models.\@ Both are asymptotically equivalent to cross validation,
  although LOOIC has been shown to outperform in WAIC in certain cases, such as
  uninformative priors or when influential observations are present in the
  data. \cite{watanabe_higher_2018, vehtari_practical_2017}

  Samples from the posterior predictive distribution (PPD) are useful tool
  for diagnosing lack of fit. In order to obtain such samples, the parameter
  vectors \(\theta^{(1)}, \ldots, \theta^{(N_{\mathrm{iter}})}\) are used to
  generate samples \(y_{\mathrm{rep}}^{(1)}, \ldots, y_{\mathrm{rep}}^{(N_{
  \mathrm{iter}})}\) such that \(y_{\mathrm{rep}}^{(m)} \sim \pi(y \mid
  \theta^{(m)})\). The samples are standardised and used in posterior
  predictive checks, where summary statistics of the standardised data is
  compared to the distribution formed by the PPD samples with respect to
  said statistic. If the model fits the data well, one would expect
  the distribution to be close to the observed statistic.

  \subsection{Overview of data}
  The raw data originate from a weather station in the south of England and
  consist of \(48212\) measurements of cumulative precipitation
  in millimetres over a 24 hour period, from 12:00 to 12:00 the day after,
  spanning approximately 132 years. Tail observations are removed so that the
  data are exactly over a \(N =132\) year period, where the measurements in
  year \(n\) will be denoted by \(X_1^{(n)}, \ldots, X_{365}^{(n)}\) with \(n
  \in \{1, \ldots, 365\}\). The maximum annual precipitation is then defined by
  \(y^{(n)} = \max\{X_1^{(n)}, \ldots, X_{365}^{(n)}\}\). This is used to
  define the data vector \(y = \{y^{(1)}, \ldots, y^{(N)}\}\) which constitutes
  the data used in the study. Table \ref{tab:data_summary} shows summary
  statistics for \(y\).

  \begin{table}[H]
    \centering
    \caption{Summary statistics of the aggregated data, \(y\).}
    \begin{tabular}{cccccc}
      \toprule
      Min.  &  \(Q_1\) &  Median  &   Mean   &  \(Q_3\)  &  Max. \\
      \midrule
      10.35 &  22.82   &  27.95   &   29.88  &  36.12    &  64.22  \\
      \bottomrule
    \end{tabular}\label{tab:data_summary}
  \end{table}

  \begin{figure}[H]
    \centering
    \includegraphics[width = 0.96\textwidth]{trendplot.pdf}
    \caption{Line plot of the aggregated data, \(y\).}
  \end{figure}

  \section{Bayesian modelling framework}\label{bayesian_modelling_framework}
  The simplest model is a fixed-parameter model, such that the response is
  assigned \(\mathrm{GEV}(\mu, \sigma, \xi)\) likelihood where \(\mu \sim
  \mathrm N(0, \sigma_\mu)\), \(\sigma \sim \mathrm{Exp}(\lambda_\sigma)\),
  and \(\xi\) is assigned a \(\mathrm{Beta}(\alpha_\xi, \beta_\xi)\) prior
  which is translated onto the \([-0.5, 0.5]\) interval.\@ Furthermore, \(\sigma
  _\mu = 10\), \(\lambda_\sigma = 3\), and \(\alpha_\xi = \beta_\xi = 4\) so
  that the parameter \(\xi\) has most density close to the origin, favouring
  small values of \(\xi\), in the spirit of \cite{johannesson_approximate_2022}.
  \@ For non-stationary modelling, two hierarchical models, in which the
  location parameter \(\mu\) evolves with time, are presented.\@ In the former
  such model, \(\mu_t = \mu_0 (1 + \Delta(t - t_c))\), where \(\mu_0 \sim
  \mathrm N(0, \sigma_ {\mu_0})\), \(\Delta \sim \mathrm N(0, \sigma_ {\Delta}
  )\) and \(\sigma_{\mu_0} = \sigma_ \Delta = 10\), and \(t_c = 66\) is the
  midpoint of elapsed time in the data set.

  In the latter model, the location parameter \(\mu_t\) at time \(t\) is
  modelled using a piecewise linear trend \(\mu_t = \mu_0 + \sum_{j = 1}^m
  \beta_j \mathbf 1_{\{t > t_j\}} (t - t_j)\) where \(1 = t_1 < \cdots < t_m < 
  n\) is an equally-spaced partition of \(\{1, \ldots, n\}\) into \(m\)
  timepoints. As before, \(\sigma \sim \mathrm{Exp}(\lambda_\sigma)\),
  \(\mu_0 \sim \mathrm N(0, \sigma_{\mu_0})\), and \(\xi \sim \mathrm{Beta}
  (\alpha_\xi, \beta_\xi)\) shifted onto the \([-0.5, 0.5]\) interval.\@ Three
  prior structures for the trend coefficients \(\beta_1, \ldots, \beta_m\) are
  considered:
  \begin{enumerate}[label = (\alph*)]
    \item an i.i.d.\@ prior where \(\beta_1 \sim \mathrm N(0,
      \sigma_\beta^{(1)})\) and \(\beta_j \sim \mathrm N(0, \sigma_\beta)\)
      for \(j \geq 2\),
    \item a random walk prior \(\beta_1 \sim \mathrm N(0, \sigma_\beta^{(1)})
      \) and \(\beta_j \sim \mathrm N(\beta_{j - 1}, \sigma_\beta)\) for
      \(j \geq 2\), and
    \item an AR(1) prior where \(\beta_1 \sim \mathrm N(0, \sigma_\beta^{(1)})
      \), and \(\beta_j \sim \mathrm N(\phi \beta_{j - 1}, \sigma_\beta)\)
      for \(j \geq 2\).
  \end{enumerate}

  Here, \(\sigma^{(1)}_\beta = 0.10\) while \(\sigma_\beta = 0.01\) to allow
  the trend to adapt to greater variations when the data exhibits a rapid rise
  or fall over time. In the AR(1) prior, \(\phi\) is assigned a
  \(\mathrm{Unf}(-1, 1)\) prior to ensure stationarity.\@ For each prior
  structure, the number of coefficients \(m\) is varied and optimised over
  WAIC and LOOIC to select the optimal architecture for the latent layer.

  \section{Results}
  All of the fitted models converged to their respective stationary
  distributions with \(\hat R \approx 1\) and \(\widehat{\mathrm{ESS}}\)
  exceeding a hundred for all parameters.\@ Figures 6-11 in Appendix A show
  trace and autocorrelation for the models, with the random walk prior
  representing the piecewise model. Noise-like behaviour observed in the trace
  plots is indicative of convergence, and a steep drop in the autocorrelation
  function \(\rho(l)\) for \(l > 5\) indicates independent sampling.

  Table \ref{tab:waic_looic} shows LOOIC and WAIC estimates for each of the
  proposed models along with the optimal value of \(m\).\@ The piecewise model
  with the random walk prior resulted in the most favourable results with
  respect to WAIC and LOOIC despite relatively small differences in
  performance. However, the non-stationary models display a marked increase in
  accuracy compared to the stationary fixed-parameter model (\(\widehat{
  \mathrm{LOOIC}} \approx 965\), \(\widehat{\mathrm{WAIC}} \approx 962\)). A
  line plot of WAIC and LOOIC estimates for the proposed models are shown in
  Figure \ref{fig:waic_looic}.\@ The optimal value of the hyperparameter \(m\)
  in the piecewise linear trend model is \(m = 18\). In light of these results,
  the remainder of the model evaluation will focus on the random walk piecewise
  model with said value of \(m\). Summary statistics for parameters \(\mu_0\),
  \(\sigma\), and \(\xi\) is shown in Table \ref{tab:rw_posterior}, and an index
  plot of the \(\beta_j\) with 95\% credible sets is shown in Figure
  \ref{fig:betas}.

  \begin{table}[H]
    \centering
    \caption{LOOIC and WAIC estimates for the proposed models.\@ For the
      piecewise linear trend model, estimates for all three prior structures
      are shown for the optimal choice of hyperparameter \(m\), shown enclosed
      in paragraphs alongside information criteria estimates.}
    \begin{tabular}{rccccc}
      \toprule
      &            &                           & \multicolumn{3}{c}{Piecewise trend}     \\
                                                          \cmidrule(rl){4-6}
                   & Stationary & Linear trend & AR(1)       & I.I.D.      & R.W.        \\
      \midrule
      \(\widehat{\mathrm{LOOIC}}\) & 965.2      & 955.0        & 954.2 (6)  & 953.8 (18)   & 953.2 (18)  \\
      \(\widehat{\mathrm{WAIC}}\)  & 961.9      & 951.9        & 951.1 (4)  & 951.0 (33)   & 950.3 (18)  \\
      \bottomrule
    \end{tabular}
    \label{tab:waic_looic}
  \end{table}

  \begin{figure}[H]
    \centering
    \includegraphics[width=0.95\textwidth]{optim_m.pdf}
    \caption{Line plot showing WAIC and LOOIC estimates as a function of number
    of coefficients \(m\) in the piecewise linear trend model. Dotted lines
    show WAIC and LOOIC for the linear trend model.}
    \label{fig:waic_looic}
  \end{figure}

  The comparison of the quantiles of the standardised observed data with
  standardised PPD samples, shown in Figure \ref{fig:qqplots}, along with
  comparisons of empirical density function estimates, shown in Figure
  \ref{fig:edf_comparison}, indicate that the random walk piecewise model is a
  good fit for the data. Posterior predictive checks shown in Figure
  \ref{fig:ppc_rw} with Bayes \(p\)-values displayed with each summary statistic
  show that the model is able to adapt to the data reasonably well to the data,
  with the maximum, mean and median values observed in the data falling within
  the 95\% credible sets of the transformed PPD distribution. The minimum
  deviates the most, with a Bayes \(p\)-value of \(0.14\), which is far from
  being indicative of a lack of fit, although it suggests that the model
  slightly overestimates the minimum in the data. A comparison of density
  function estimates of a subset of samples from the standardised PPD, shown in
  Figure \ref{fig:edf_comparison} further reinforces the aforementioned result,
  showing that the curves align closely.

  That the random walk prior structure is most effective in modelling the trend
  may be explained by the smoothness condition that is imposed on the \(\beta_j
  \) compared to the i.i.d.\@ prior, where the coefficients are allowed to
  explore the sample space more freely. However, it seems that adding an
  autoregressive coefficient as in the AR(1) is detrimental in terms of
  accuracy, which could indicate that the added hyperparameter may be
  overfitting the data.

  \begin{table}[H]
    \centering
    \caption{Summary statistics for fitted parameters in the random walk
    piecewise model. The quantities \(q_{0.025}\) and \(q_{0.975}\) are 
    the \(2.5\%\) and \(97.5\%\) quantiles, respectively. Thus, the \(95\%\)
    credible sets for the parameters are given by \((q_{0.025}, q_{0.975})\).}
    \begin{tabular}{lccccc}
      \toprule
                     &  Mean     &  Median   & SD     & \(q_{0.025}\) &  \(q_{0.975}\) \\
      \midrule
      \(\mu_0\)      & 20.7      & 20.7      & 1.34   & 18.0          & 23.3           \\
      \(\sigma\)     & 6.56      &  6.54     & 0.454  & 5.72          & 7.50           \\
      \(\xi\)        & 0.0593    &  0.0574   & 0.0629 & -0.0587       & 0.187          \\
      \bottomrule
    \end{tabular}
    \label{tab:rw_posterior}
  \end{table}

  \newpage

  \vfill
  \begin{figure}[h!]
    \centering
    \includegraphics[width = \textwidth]{qqplot_yrep.pdf}
    \caption{Q-Q plot of standardised observed data against
      standardised PPD samples.}
    \label{fig:qqplots}
  \end{figure}
  \vfill
  \begin{figure}[h!]
    \centering
    \includegraphics[width = 0.95\textwidth]{ppc_edf_overlay.pdf}
    \caption{(Left) Densities of a subset of standardised PPD samples shown in
      light blue in comparison to density estimate of standardised observed
      data, shown in darker blue. (Right) Empirical cumulative density function
      of a subset of standardised PPD samples in light blue compared to the ECDF
      of standardised observed data \(y\) in dark blue.}
    \label{fig:edf_comparison}
  \end{figure}
  \vfill

  \newpage

  \vfill
  \begin{figure}[h!]
    \centering
    \includegraphics[width = 0.95\textwidth]{ppc_rw.pdf} 
    \caption{Posterior predictive checks for the random walk piecewise 
    model.\@ The histogram shows estimates from the standardised posterior
    predictive sample matrix, and the vertical blue line represents the observed
    estimate from the data. Dashed lines show 95\% probability bands. The
    quantity \(p\) represents the cumulative density evaluated at the estimated
    statistic, called the Bayes \(p\)-value.}
    \label{fig:ppc_rw}
  \end{figure}
  \vfill
  \begin{figure}[h!]
    \centering
    \includegraphics[width = 0.95\textwidth]{betas.pdf}
    \caption{Index plot of coefficients \(\beta_1, \ldots, \beta_m\), showing
    their posterior mean with 95\% credible sets.}
    \label{fig:betas}
  \end{figure}
  \vfill

  \newpage

  \begin{figure}[h!]
    \centering
    \includegraphics[width = \textwidth]{post_mean.pdf}
    \caption{Plot comparing posterior mean of GEV distribution with location
    parameter \(\mu_t = \tau(t, \theta)\) with line plot of data.\@ Here,
    \(\theta\) is a parameter vector modelled in the latent layer. Dashed lines
    indicate 95\% probability bands.}
    \label{fig:post_mean}
  \end{figure}


  \section{Discussion}
  Model comparison using information criteria clearly favoured non-stationary
  models with a temporally evolving location parameter.\@ Although differences
  in accuracy between variants of the piecewise model and the linear time
  trend model are small, the random walk model is most favourable when
  considering the LOOIC and WAIC information criteria. Empirical model
  evaluation using posterior predictive sampling showed that the model describes
  variability within the data reasonably well, with quantiles generated from the
  posterior predictive distribution and the data falling approximately on the
  identity line on a Q-Q plot, indicating the absence of lack of fit for the
  optimal model.

  Improved accuracy in models that implement an evolving location parameter
  over time suggests that the mean annual maximum precipitation is variable
  with time.\@ A line plot of the posterior mean of the GEV variable modelled
  in the selected piecewise model shows an upward trend from \(t = 0\) up to
  approximately one hundred years, followed by what seems to be a downward
  trend in the remaining thirty or so years of the data. Whether this trend
  will continue in this fashion remains to be seen.\@ However, it is important
  to note that the trend observed in these data may not reflect what one would
  observe in other locations.

  The modelling of extremes such as those explored in these data is an
  interesting topic in statistics, both in theory and practice.\@ Limiting
  behaviours of block maxima ensured by the Fisher-Tippett theorem make the
  GEV distribution an easy candidate for the likelihood of data, but as shown
  in this article, the design of the latent layer in a Bayesian setting can
  vastly improve the accuracy of these models, such as in an iterated 
  setting with parameters evolving over time. Additionally, the addition of
  spacial effects and the inclusion of covariates relevant to the measuring
  stations from which the data are sources can improve accuracy even
  further, e.g. in \cite{dyrrdal_bayesian_2015}, \cite{hazra_bayesian_2023},
  and \cite{johannesson_approximate_2022}. By leveraging these advanced
  techniques, more precise estimates and better predictions of extreme events
  can be achieved.

  \printbibliography

  \newpage

  \appendix

  \section{Convergence of MCMC samplers}
  \subsection{Stationary model}
  \begin{figure}[H]
    \centering
    \includegraphics[width = 0.85\textwidth]{trace_null.pdf}
    \caption{Trace plot of the parameters for the stationary (fixed-parameter)
    model.}
  \end{figure}
  \begin{figure}[H]
    \centering
    \includegraphics[width = 0.85\textwidth]{acf_null.pdf}
    \caption{Plot of autocorrelation function \(\rho(l)\) for the
    stationary model.}
  \end{figure}

  \subsection{Non-stationary model (linear trend)}
  \begin{figure}[H]
    \centering
    \includegraphics[width = 0.85\textwidth]{trace_timetrend.pdf}
		\caption{Parameter summary statistics for the timetrend model.}
  \end{figure}
  \begin{figure}[H]
    \centering
    \includegraphics[width = 0.76\textwidth]{acf_timetrend.pdf}
    \caption{Plot of autocorrelation function \(\rho(l)\) for
    the linear time trend model.}
  \end{figure}

  \subsection{Non-stationary model (non-linear, random walk prior)}

  \begin{figure}[H]
    \centering
    \includegraphics[width = 0.7\textwidth]{trace_rw.pdf}
    \caption{Trace plot of parameters of random walk prior piecewise 
    model.}
  \end{figure}

  \begin{figure}[H]
    \centering
    \includegraphics[width = 0.6\textwidth]{acf_rw_1.pdf}
    \includegraphics[width = 0.6\textwidth]{acf_rw_2.pdf}
    \caption{Plot of autocorrelation function \(\rho(l)\) for
    the random walk prior piecewise model.}
  \end{figure}

\end{document}
