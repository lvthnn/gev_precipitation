\documentclass[10pt, a4paper]{paper}

\usepackage[T1]{fontenc}
\usepackage[utf8]{inputenc}

\usepackage[scale = 0.75]{geometry}

\usepackage[UKenglish]{babel}

\usepackage{amsmath, amsfonts, amssymb}
\usepackage[scale=0.8]{noto-mono}
\usepackage{bm}

\DeclareMathAlphabet{\mathcal}{OMS}{cmsy}{m}{n}

\usepackage{graphicx}
\usepackage{tabularx}
\usepackage{caption}
\usepackage{cleveref}
\usepackage{float}

\usepackage{booktabs}

% \renewenvironment{abstract}
%  {\small
%   \begin{center}
%   \bfseries \abstractname\vspace{-.5em}\vspace{0pt}
%   \end{center}
%   \list{}{%
%     \setlength{\leftmargin}{20mm}
%     \setlength{\rightmargin}{\leftmargin}%
%   }%
%   \item\relax}
%  {\endlist}

\captionsetup{
  labelfont = bf,
  labelsep = period,
  font = small
}

\title{Application of hierarchical Bayesian models
       to extreme precipitation events}
\author{Kári Hlynsson}
\date{}

\begin{document}
  
  \maketitle

  \section{Introduction}
  \begin{itemize}
    \item Describe the pretext of the project; application to protection
      of infrastructure, prevention of disasters etc.
    \item Why is the Bayesian framework useful for the modelling of
      precipitation phenomena?
    \item What are the prior attempts of modelling, how are they modelled.
      Is this project a novel approach in any way?
  \end{itemize}

  \section{Preliminaries}
  \subsection{Theoretical background}
  \begin{itemize}
    \item The generalised extreme value distribution
      \begin{itemize}
        \item Parameters
        \item Support, cover for different values of \(\xi\)
        \item Describe usual ranges of parameters in the context of
          environmental modelling, especially precipitation -- cf. literature
      \end{itemize}
    \item Fisher-Tippett theorem states that block maxima of an i.i.d.\@
      collection of random variables converges to the GEV distribution
    \item Return levels?
  \end{itemize}
  \subsection{Markov-Chain Monte Carlo (MCMC) sampling and \texttt{stan}}
  \begin{itemize}
    \item Describe the general idea underlying MCMC sampling, i.e.\@ the
      high-dimensional integral setting involved in estimation
    \item Hamiltonian Monte Carlo and the NUTS algorithm.
  \end{itemize}
  \subsection{Model diagnostics and evaluation}
  \begin{itemize}
    \item Talk about \(\widehat{\mathrm{ESS}}\), \(\hat R\), autocorrelation
      and trace plots.
    \item Mention how I will evaluate convergence of chains.
    \item Describe LOO, PSIS, and WAIC 
  \end{itemize}
  \section{Bayesian modelling framework}
  \subsection{Fixed parameter models}
  \begin{itemize}
    \item Null model \(Y_n \sim \mathrm{GEV}(\mu, \sigma, \xi)\) where
      \(\mu \sim \mathcal N(0, \sigma_\mu^2)\), \(\sigma \sim \mathrm{Exp}
      (\lambda_\sigma)\), and \(\xi \sim \mathrm{Beta}(\alpha_\xi, \beta_\xi)\)
      transformed onto the interval \([-0.5, 0.5]\).
  \end{itemize}
  \subsection{Timetrend model}
  \begin{itemize}
    \item Timetrend model \(Y_n \sim \mathrm{GEV}(\mu_n, \sigma, \xi)\) where
      \(\mu_n = \mu_0\big(1 + \Delta(t - t_c)\big)\), \(\Delta \sim
      \mathcal N(0, \sigma_\Delta^2)\), \(\sigma \sim
      \mathrm{Exp}(\lambda_\sigma)\) and \(\xi \sim \mathrm{Beta}(\alpha_\xi,
      \beta_\xi)\)
      shifted onto the interval \([-0.5, 0.5]\). \(\Omega\)
  \end{itemize}
  \subsection{Hierarchical model}
  \begin{itemize}
    \item Split the time interval \(\{1, \ldots, T\}\) into \(m\) evenly
      spaced points excluding time \(T\), \(1 = t_1 < t_2 < \ldots < t_m < T\).
    \item We let \(\mu_t = \sum_{j = 1}^k \beta_j(t - t_j) \mathbf
      1_{\{t \geq t_j\}}\), assign priors \(\sigma \sim
      \mathrm{Exp}(\lambda_\sigma)\) and \(\xi \sim \mathrm{Beta}(\alpha_\xi,
      \beta_\xi)\) shiften onto the interval \([-0.5, 0.5]\).
    \item Three priors for the vector \(\boldsymbol\beta\) are compared,
      namely
      \begin{itemize}
        \item an i.i.d.\@ Gaussian noise prior, \(\beta_j \sim \mathcal N(0,
          \sigma_\varepsilon^2)\),
        \item a random walk prior, \(\beta_1 \sim \mathcal N(0, \sigma_\beta)\)
          and \(\beta_j \sim \mathcal N(\beta_{j - 1})\)
        \item an AR1 prior where \(\beta_1 \sim \mathcal N(0, \sigma_\beta)\)
          and \(\beta_j \sim \mathcal N(\alpha + \phi \beta_{j - 1}, \sigma_\varepsilon)\)
          for \(j \geq 2\)
      \end{itemize}
    \item We then optimize over the hyperparameter \(m\) to obtain the best
      results
  \end{itemize}
  \section{Results}
  \begin{itemize}
    \item Mention convergence good excluding some divergent transitions
      (proportion less than 3\%), trace plots are shown in appendix
    \item \(\hat R \approx 1\), \(\widehat{\mathrm{ESS}} > N\)
    \item Table with parameter estimate tables for each of the parameters
    \item Table with WAIC, LOO, and PSIS for each of the models side by
      side with \(m = 6\) as candidate
    \item Table showing WAIC, LOO, and PSIS and a figure showing trends of
      these estimates as a function of \(m \in [n]\)
  \end{itemize}

  \begin{table}[H]
    \centering
    \caption{Parameter summary statistics for the null model.}
    \begin{tabular}{crrrrrrcc}
      \toprule
      & Mean & Mean SE & SD & 2.5\% & 50\% & 97.5\% & \(\widehat{\mathrm{ESS}}\) \\
      \midrule
      \(\mu\) & 25.3575 & 0.0046 & 0.7484 & 23.9136 & 25.3489 & 26.8523 & 26294\\
      \(\sigma\) & 7.7143 & 0.0033 & 0.5395 & 6.7254 & 7.6897 & 8.8425 & 26309\\
      \(\xi\) & \(-0.0106\) & 0.0004 & 0.0604 & \(-0.1222\) & \(-0.0130\) & 0.1134 & 23787\\
      \bottomrule
    \end{tabular}
  \end{table}

  \begin{table}[H]
    \centering
    \caption{Parameter summary statistics for the timetrend model.}
    \begin{tabular}{crrrrrrc}
      \toprule
      & Mean & Mean SE & SD & 2.5\% & 50\% & 97.5\% & \(\widehat{\mathrm{ESS}}\) \\
      \midrule
      \(\mu_0\) & 25.3093 & 0.0053 & 0.7164 & 23.9188 & 25.2998 & 26.7283 & 18283\\
      \(\sigma\) & 7.2481 & 0.0041 & 0.5370 & 6.2669 & 7.2217 & 8.3736 & 17497 \\
			\(\xi\) & 0.0267 & 0.0005 & 0.0670 & -0.0979 & 0.0247 & 0.1630 & 17907 \\
      \(\Delta\) & 0.0022 & 0.0000 & 0.0006 & 0.0009 & 0.0021 & 0.0034 & 49770 \\
      \bottomrule
    \end{tabular}
  \end{table}

  \begin{table}[H]
		\centering
		\caption{Parameter summary statistics for the i.i.d.\@ hierarchical model.}
    \begin{tabular}{crrrrrrcc}
     \toprule
			& Mean & Mean SE & SD & 2.5\% & 50\% & 97.5\% & \(\widehat{\mathrm{ESS}}\) \\
			\midrule
			\(\beta_1\) & 2.2298 & 0.0018 & 0.2277 & 1.7759 & 2.2336 & 2.6718 & 15951 \\
			\(\beta_2\) & \(-3.2579\) & 0.0049 & 0.5961 & \(-4.4262\) & \(-3.2634\) & \(-2.0913\) & 14792 \\
			\(\beta_3\) & 1.7557 & 0.0052 & 0.6593 & 0.5108 & 1.7382 & 3.0974 & 16318 \\
			\(\beta_4\) & \(-1.1203\) & 0.0045 & 0.6225 & \(-2.3156\) & \(-1.1303\) & 0.1306 & 19194 \\
			\(\beta_5\) & 0.6396 & 0.0045 & 0.6276 & \(-0.5978\) & 0.6374 & 1.8775 & 19537 \\
			\(\beta_6\) & \(-0.2340\) & 0.0043 & 0.5988 & \(-1.3858\) & \(-0.2476\) & 0.9747 & 19817 \\
			\(\beta_7\) & 0.2380 & 0.0042 & 0.6065 & \(-0.9291\) & 0.2269 & 1.4593 & 20673 \\
			\(\beta_8\) & \(-0.8775\) & 0.0042 & 0.6611 & \(-2.1926\) & \(-0.8649\) & 0.3987 & 24304 \\
			\(\sigma\) & 9.0816 & 0.0058 & 0.8057 & 7.6243 & 9.0454 & 10.7854 & 19423 \\
			\(\tau\) & 2.0024 & 0.0096 & 2.0019 & 0.0482 & 1.4161 & 7.3918 & 43555 \\
			\(\xi\) & 0.0019 & 0.0006 & 0.0828 & \(-0.1502\) & \(-0.0017\) & 0.1730 & 16960 \\
			\bottomrule
    \end{tabular} 
  \end{table}

  \begin{table}[H]
		\centering
		\caption{Parameter summary statistics for the random walk hierarchical model.}
    \begin{tabular}{crrrrrrc}
     \toprule
			& Mean & Mean SE & SD & 2.5\% & 50\% & 97.5\% & \(\widehat{\mathrm{ESS}}\) \\
			\midrule
			\(\beta_1\) & 1.8779 & 0.0026 & 0.2753 & 1.3014 & 1.8884 & 2.3879 & 11368 \\
			\(\beta_2\) & -2.2532 & 0.0066 & 0.6797 & -3.5642 & -2.2665 & -0.8758 & 10743\\
			\(\beta_3\) & 0.7112 & 0.0059 & 0.6394 & -0.4595 & 0.6915 & 2.0363 & 11680 \\
			\(\beta_4\) & -0.5104 & 0.0042 & 0.5293 & -1.5728 & -0.4988 & 0.5064 & 16151 \\
			\(\beta_5\) & 0.3131 & 0.0038 & 0.4908 & -0.6317 & 0.3068 & 1.2964 & 16702 \\
			\(\beta_6\) & -0.0404 & 0.0035 & 0.4682 & -0.9591 & -0.0443 & 0.8986 & 17806 \\
			\(\beta_7\) & 0.0611 & 0.0036 & 0.4786 & -0.8490 & 0.0471 & 1.0383 & 17755 \\
			\(\beta_8\) & -0.6947 & 0.0039 & 0.5661 & -1.8410 & -0.6806 & 0.3889 & 21513 \\
			\(\sigma\) & 8.8751 & 0.0057 & 0.8147 & 7.3887 & 8.8358 & 10.5877 & 20115 \\
			\(\tau\) & 2.3623 & 0.0078 & 0.9292 & 0.9751 & 2.2162 & 4.5857 & 14083 \\
			\(\xi\) & 0.0421 & 0.0007 & 0.0896 & -0.1244 & 0.0388 & 0.2261 & 16188 \\
			\bottomrule
    \end{tabular} 
  \end{table}

  \begin{table}[H]
		\centering
		\caption{Parameter summary statistics for the AR1 hierarchical model.}
    \begin{tabular}{crrrrrrc}
     \toprule
			& Mean & Mean SE & SD & 2.5\% & 50\% & 97.5\% & \(\widehat{\mathrm{ESS}}\) \\
			\midrule
      beta[1] & 1.9853 & 0.0021 & 0.2310 & 1.5324 & 1.9853 & 2.4421 & 12600\\
      beta[2] & -2.6416 & 0.0052 & 0.5657 & -3.7952 & -2.6304 & -1.5543 & 11657\\
      beta[3] & 1.2968 & 0.0051 & 0.5710 & 0.2223 & 1.2777 & 2.4655 & 12321\\
      beta[4] & -1.0662 & 0.0044 & 0.5231 & -2.0865 & -1.0760 & -0.0197 & 14002\\
      beta[5] & 0.7417 & 0.0044 & 0.5170 & -0.2772 & 0.7404 & 1.7540 & 14093\\
      beta[6] & -0.3992 & 0.0042 & 0.5014 & -1.3911 & -0.4043 & 0.5987 & 14351\\
      beta[7] & 0.3087 & 0.0042 & 0.5227 & -0.6862 & 0.2950 & 1.3778 & 15596\\
      beta[8] & -0.7093 & 0.0040 & 0.5497 & -1.8376 & -0.6958 & 0.3341 & 19044\\
      sigma & 8.7397 & 0.0064 & 0.7942 & 7.3026 & 8.6966 & 10.4252 & 15177\\
      tau & 0.8126 & 0.0030 & 0.3545 & 0.3718 & 0.7363 & 1.6891 & 14176\\
      phi & -0.7695 & 0.0020 & 0.2454 & -1.2245 & -0.7825 & -0.2521 & 15359\\
      xi & 0.0491 & 0.0008 & 0.0897 & -0.1168 & 0.0456 & 0.2329 & 11860\\
			\bottomrule
    \end{tabular} 
  \end{table}

  \section{Discussion}
  \begin{itemize}
    \item Figure of predicted values with 95\% credible sets against observed
      values for best model
  \end{itemize}

  \newpage

  \appendix

  \section{Convergence of MCMC samplers}

  \section{Code used in the project}

\end{document}
